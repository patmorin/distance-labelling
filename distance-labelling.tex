\documentclass{patmorin}
\setlength{\parskip}{1ex}
\listfiles
\usepackage{pat}
\usepackage{paralist}
\usepackage{dsfont}  % for \mathds{A}
\usepackage[utf8x]{inputenc}
\usepackage{skull}
\usepackage{paralist}
\usepackage{graphicx}
\usepackage[noend]{algorithmic}

\usepackage[normalem]{ulem}
\usepackage{cancel}
\usepackage{enumitem}

\usepackage{todonotes}

\usepackage[longnamesfirst,numbers,sort&compress]{natbib}

% \usepackage[mathlines]{lineno}
% \setlength{\linenumbersep}{2em}
% \linenumbers
% \rightlinenumbers
% \linenumbers
% \newcommand*\patchAmsMathEnvironmentForLineno[1]{%
%  \expandafter\let\csname old#1\expandafter\endcsname\csname #1\endcsname
%  \expandafter\let\csname oldend#1\expandafter\endcsname\csname end#1\endcsname
%  \renewenvironment{#1}%
%     {\linenomath\csname old#1\endcsname}%
%     {\csname oldend#1\endcsname\endlinenomath}}%
% \newcommand*\patchBothAmsMathEnvironmentsForLineno[1]{%
%  \patchAmsMathEnvironmentForLineno{#1}%
%  \patchAmsMathEnvironmentForLineno{#1*}}%
% \AtBeginDocument{%
% \patchBothAmsMathEnvironmentsForLineno{equation}%
% \patchBothAmsMathEnvironmentsForLineno{align}%
% \patchBothAmsMathEnvironmentsForLineno{flalign}%
% \patchBothAmsMathEnvironmentsForLineno{alignat}%
% \patchBothAmsMathEnvironmentsForLineno{gather}%
% \patchBothAmsMathEnvironmentsForLineno{multline}%
% }


\newcommand{\coloured}[2]{{\color{#1}{#2}}}
\newenvironment{colourblock}[1]{\color{#1}}{}

% Taken from
% https://tex.stackexchange.com/questions/42726/align-but-show-one-equation-number-at-the-end
\newcommand\numberthis{\addtocounter{equation}{1}\tag{\theequation}}




\title{\MakeUppercase{Distance Labelling of Unweighted Planar Graphs}\thanks{This research was partly funded by NSERC.}}
\author{Pat Morin%
    \thanks{School of Computer Science, Carleton University}}


\DeclareMathOperator{\dist}{dist}
\DeclareMathOperator{\depth}{depth}
\DeclareMathOperator{\polylog}{polylog}
\DeclareMathOperator{\diam}{diam}

\newcommand{\comparable}{\mathbin{\diamond}}


\newcommand{\colored}[2]{{\color{#1}#2}}

\usepackage{tabularx}


\begin{document}

% \begin{titlepage}
\maketitle

\begin{abstract}
    We give a \emph{distance labelling scheme} for unweighted undirected planar graphs in which each vertex is assigned a label of length $O(n^{1/3}\polylog n)$ such that the labels of any two vertices are sufficient to compute the distance between them.
\end{abstract}

\section{Introduction}

A class of $\mathcal{G}$ of graphs has an $f(n)$-bit \emph{distance labelling scheme} if there exists a function $D:\{0,1\}^*\to\N$ such that, for each $n$-vertex graph $G\in\mathcal{G}$ there exists a \emph{labelling} $\varphi_G:V(G)\to\{0,1\}^{f(n)}$ such that $D(\varphi_G(v),\varphi_G(w)) = \dist_G(v,w)$ for each pair of vertices $v,w\in V(G)$.

\begin{thm}\label{main}
    The class of planar graphs has an $O(n^{1/3}\polylog n)$-bit distance labelling scheme.
\end{thm}

\section{The Main Tool}
\label{main_tool}

Let $G$ be a graph.\footnote{All graphs are connected, finite, undirected, simple, and unweighted, unless specified otherwise}  For any $x,y\in V(G)$, let $\dist_G(x,y)$ denote the minimum length of a path in $G$ with endpoints $x$ and $y$.  Any path $x_1,\ldots,x_m$ of length $\dist_G(x_1,x_m)$ is called an $(x_1,x_m)$ \emph{$G$-geodesic}.

Let $T$ be a rooted tree.  A node $x\in V(T)$ is a \emph{$T$-ancestor} of $y\in V(T)$ if $x=y$ or $x$ is a $T$-ancestor of the parent of $y$.  If $x$ is a $T$-ancestor of $y$, then $y$ is a \emph{$T$-descendant} of $x$.  Note that every node of $T$ is both a $T$-ancestor and $T$-descendant of itself.  We use $\preceq_T$ to denote the $T$-ancestor relationship.  If $x\prec_T y$ and $x\neq y$ then we say that $x$ is \emph{strict} $T$-ancestor of $y$, $y$ is a \emph{strict} $T$-descendant of $x$ and $\prec_T$ denotes the partial order over $V(T)$ defined by the strict $T$-ancestor relation.  Define the relation $\comparable_T$ such that $x\comparable_T y$ if and only if $x\preceq_T y$ or $y\preceq_T x$.

For two nodes $x,y\in V(T)$, $P_T(x,y)$ denotes the unique path from $x$ to $y$ in $T$.  A path $v_1,\ldots,v_{m}$ in a rooted tree $T$ is \emph{$T$-downward} $v_1\preceq_T v_m$, is \emph{$T$-upward} if $v_m\preceq_T v_1$ and is \emph{$T$-vertical} if it is upward or downward.



\subsection{When $v_1$ and $w_1$ are on the same face}

Let $G$ be a plane graph, let $T$ be a BFS tree of $G$, and let $v_1,\ldots,v_s$ and $w_1,\ldots,w_t$ be disjoint upward paths in $T$ with $v_1$ and $w_1$ on the same face of $G$.  We are interested labelling $v_1,\ldots,v_s$ and $w_1,\ldots,w_t$ so that we can recover $\dist_G(v_i,w_j)$ from the labels of $v_i$ and $w_j$ for any $i,j\in\{1,\ldots,s\}\times\{1,\ldots,t\}$.


% First observe that this is trivial to achieve using $O(\log n)$-bit labels if  $v_1\comparable_T w_1$ since then all vertices are contained in a single vertical path in $T$.  Since $T$ is a BFS tree, this implies that $\dist_G(v_i,w_j)=|\depth_T(v_i)-\depth_T(w_j)|$.
%
% Now we consider the interesting case in which neither of $v_1$ or $w_1$ is a $T$-ancestor of the other.

Let $a$ be the lowest-common $T$-ancestor of $v_s$ and $w_t$.  Since $v_1$ and $w_1$ are on the same face of $G$, there exists a Jordan curve $f$ that contains every edge of $P_T(v_1,w_1)$, but does not intersect the interior of any other edge of $G$.  Let $f_0$ and $f_1$ be the two components of $\R^2\setminus f$.  For each $b\in\{0,1\}$, let $G_b$ be the subgraph of $G$ containing every edge and vertex of $G$ contained in $f\cup f_b$.

\begin{obs}\label{in_out}
    For any $i,j\in\{1,\ldots,s\}\times\{1,\ldots,t\}$,  $\dist_G(v_i,w_j)=\min\{\dist_{G_b}(v_i,w_j): b\in\{0,1\}\}$.
\end{obs}

\begin{proof}
    Let $a$ be the lowest common $T$-ancestor of $v_1$ and $w_1$. If $P$ is a $G$-geodesic that begins at $v_i$ and ends at $w_j$ that contains a vertex $x\in f_0$ and a vertex $z\in f_1$, then $P$ contains a vertex $y\in f$.  If $y\in V(P_T(a,v_1))$, then portion of $P$ from $v_i$ to $y$ can be replaced with $P_T(v_i,y)$ without increasing its length.  If $y\in V(P_T(a,w_1))$ then the portion of $P$ from $y$ to $w_j$ can be replaced with $P_T(y,w_j)$ without increasing its length.  In either case, the resulting geodesic has fewer vertices in $f_0\cup f_1$. Since $P$ is finite we can therefore repeat this until $P$ has no vertices in $f_0$ or $P$ has no vertices in $f_1$.
\end{proof}

B \cref{in_out} it suffices to focus on labels that allow us to recover $\dist_{G_b}(v_i,w_j)$ for each $b\in\{0,1\}$.  For concreteness, we focus on the case $b=0$. The case $b=1$ is handled identically.

Let $M$ be the $s\times t$ distance matrix in which $M_{i,j}:=\dist_{G_0}(v_i,w_j)$ for each $i,j\in\{1,\ldots,s\}\times\{1,\ldots,t\}$.  \citet{abboud.gawrychowski.ea:near-optimal} observe that, because $v_1,\ldots,v_s$ and $w_1,\ldots,w_t$ are on a common face of $G_0$,  $M$ has the following \emph{unit Monge property}:
\begin{compactenum}[(UM1)]
    \item For each $i,j\in\{2,\ldots,s\}\times\{1,\ldots,t\}$, $M_{i,j}-M_{i-1,j} \in \{-1,0,1\}$.\label{unit_property1}
    \item For each $i,j\in\{1,\ldots,s-1\}\times\{1,\ldots,t-1\}$, $M_{i,j}+M_{i+1,j+1} \le M_{i+1,j} + M_{i,j+1}$.\label{monge_property}
\end{compactenum}

Let $S$ be the $s\times t$ matrix defined by
\begin{equation}
    S_{i,j}=\begin{cases}
                0 & \text{if $i=1$} \\
                M_{i,j}-M_{i-1,j} & \text{otherwise}
            \end{cases}  \label{s_definition}
\end{equation}
It follows from \cref{unit_property1,monge_property} that each row $S_{i,*}$ of $S$ is non-increasing and contains only entries from $\{-1,0,1\}$, i.e., $1\ge S_{i,1}\ge S_{i,2}\ge\cdots\ge S_{i,t}\ge -1$. In particular, any row of $S_i$ can be encoded in $O(\log t)$ bits by storing the index of the first $0$ and the first $-1$.  More precisely, by defining  $j_{i,0}:=\min\{0\}\cup\{j\in\{1,\ldots,t\}:S_{i,j}=0\}$ and $j_{i,-1}:=\min\{t+1\}\cup\{j\in\{1,\ldots,t\}:S_{i,j}=-1\}$ we get
\[
    S_{i,j} = \begin{cases}
        1 & \text{if $j < j_{i,0}$} \\
        0 & \text{if $j_{i,0}\le j < j_{i,-1}$} \\
        -1 & \text{if $j_{i,-1}\le j$.}
    \end{cases}
\]

Observe that, for any $1\le i_0\le i\le s$ and any $j\in\{1,\ldots,t\}$,
\begin{equation}\label{recovery}
   M_{i,j} = M_{i_0,j} + \sum_{i'=i_0+1}^i (M_{i',j}-M_{i'-1,j}) = M_{i_0,j} + \sum_{i'=i_0+1}^i S_{i,j} \enspace .
\end{equation}


This suggests a distance labelling scheme that partitions $M$ in blocks consisting of $\ell$ consecutive rows, for some integer $\ell$.  In this scheme, $w_j$ stores $M_{i_0,j}$ for each $i_0\in I := \{1,1+\ell,1+2\ell,\ldots,1+\floor{(s-1)/\ell}\}$ and $v_i$ stores $S_{i_0+1,*},\ldots,S_{i,*}$ for $i_0=\max I\cap\{1,\ldots,i\}$.  It is straightforward to check that the labels of $v_i$ and $w_j$ contain all the information needed to evaluate \cref{recovery} and recover $M_{i,j}=\dist_G(v_i,w_j)$.

With this scheme, the label of $v_i$ has length $O((i-i_0)\log t)\subseteq O(\ell\log t)\subseteq O(\ell\log n)$ and the label of $w_j$ has length $O((s/\ell)\log n)$.

\begin{lem}\label{sameface_labelling}
    Let $\ell$ be a positive integer, let $G$ be an $n$-vertex plane graph, let $T$ be a BFS tree and let $v_1,\ldots,v_s$ and $w_1,\ldots,w_t$ be two disjoint upward paths in $T$ in which $v_1$ and $w_1$ are on a common face of $G$.    Then there exists an assignment of $O((\ell + s/\ell)\log n)$-bit labels to $v_1,\ldots,v_s$ and $w_1,\ldots,w_t$ such that $\dist_G(v_i,w_j)$ can be computed from the labels of $v_i$ and $w_j$, for any $i,j\in\{1,\ldots,s\}\times\{1,\ldots,t\}$.\todo{Formalize this with the notion of partial labelling scheme.}
\end{lem}

The optimal choice of $\ell$, in this case, is $\ell=\Theta(\sqrt{s})$ which gives a scheme using labels of length $O(\sqrt{s}\log n)$.  Of course, this is not yet enough to obtain an efficient general scheme for planar graphs.  Note that there is a symmetry between the two paths $v_1,\ldots,v_s$ and $w_1,\ldots,w_t$, so we may assume that $s\le t$.


\subsection{When $v_1$ and $w_1$ are not on the same face}

Next we extend the labelling scheme in the previous section so that it works when the vertices $v_1$ and $w_1$ are not on the same face.  This extension is not immediate because there is obvious choice for the Jordan curve $f$ that defines $G_0$ and $G_1$.

A path $x_1,\ldots,x_k$ in $G$ is \emph{$T$-conforming} if, for each $1\le i< j\le k$ such that $x_i\comparable_T x_j$, the subpath $x_i,\ldots,x_j$ is $T$-vertical.

\begin{obs}\label{conforming_geodesic}
    For any graph $G$, any BFS tree $T$ of $G$, and any $x,y\in V(G)$, there exists a $T$-conforming $(x,y)$ $G$-geodesic.
\end{obs}

Let $P:=P_T(v_1,w_1)$ and let $X$ be a $T$-conforming $G$-geodesic from $v_1$ to $w_1$.  If $X=P$, then the problem is trivial, since $\dist_G(v_i,w_j)=\dist_{X}(v_i,w_j)$ and a labelling that simply numbers the vertices according to their position in $X$ is sufficient.

If $X\neq P$ then the graph $P\cup X$ has two faces $f_0$ and $f_1$ whose boundaries intersect in a cycle $f$.  Although this looks like the previous case, we are not quite done.  \cref{in_out} does not hold for this definition of $f$, $f_0$, and $f_1$.  In particular, there may exist $v_i$ and $w_j$ such that every $G$-geodesic from $v_i$ to $w_j$ contains a vertex in $f_0$ and a vertex in $f_1$.   We now examine how this happens.

By \cref{conforming_geodesic}, there exists a geodesic $P_{ij}:=x_1,\ldots,x_k$ with $x_1=v_i$ and $x_2=w_j$.  By definition $P_{ij}\cap P_T(a,v_1)$ and $P_{ij}\cap P_T(a,w_1)$ are each paths.  Using the same strategy used in the proof of \cref{in_out} we can also guarantee that $P_{ij}\cap X$ is either empty or a path.  When $P_{ij}\cap X$ is empty, $P_{ij}$ is contained in $G_b$ for some $b\in\{0,1\}$, and we have already given a labelling scheme that allows us to compute $\dist_{G_b}(v_i,w_j)$.

To handle cases where $P_{ij}\cap X$ is not empty, we define $\dist'(v_i,w_j)$ as the length of the shortest path from $v_i$ to $w_j$ that contains at least one vertex of $X$, i.e., $\dist'(v_i,w_j):=\min\{\dist_G(v_i,x)+\dist_G(w_j,x): x\in V(X)\}$.  The preceding discussion is essentially a proof of:

\begin{obs}\label{in_out2}
    For any $i,j\in\{1,\ldots,s\}\times\{1,\ldots,t\}$,  $\dist_G(v_i,w_j)=\min\{\dist'(v_i,w_j)\cup\{\dist_{G_b}(v_i,w_j): b\in\{0,1\}\}$.
\end{obs}

Let $M'$ be the distance matrix defined by $\dist'$, so that $M'_{i,j}:=\dist'(v_i,w_j)$.  The matrix $M'$ certainly has the unit property (UM\ref{unit_property1}).  It also has a the Monge property (UM\ref{monge_property}), except that the inequality is reversed:  $M'_{i,j}+M'_{i+1,j+1} \ge M'_{i+1,j}+M'_{i,j+1}$.\todo{This is not quite right, strange things happen when $X$ contains edges of $T$.}  This reversal does not prevent us from applying the same labelling scheme.  In fact, the only difference is that the matrix $S'$ defined analogously to $S$ has rows that are non-decreasing rather than non-increasing.  This proves the following result:


\begin{lem}\label{general_labelling}
    Let $\ell$ be a positive integer, let $G$ be an $n$-vertex planar graph, let $T$ be a BFS tree and let $v_1,\ldots,v_s$ and $w_1,\ldots,w_t$ be two disjoint upward paths in $T$. Then there exists an assignment of $O((\ell + s/\ell)\log n)$-bit labels to $v_1,\ldots,v_s$ and $w_1,\ldots,w_t$ such that $\dist_G(v_i,w_j)$ can be computed from the labels of $v_i$ and $w_j$, for any $i,j\in\{1,\ldots,s\}\times\{1,\ldots,t\}$.\todo{Formalize this with the notion of partial labelling scheme.}
\end{lem}

\todo[inline]{This generalizes to any two vertex disjoint geodesic.  Make a ``2-connected'' graph by joining endpoints of the input geodesics with (at most $4$) geodesic edges.  Call these geodesic edges $e_1,\ldots,e_k$.  Then consider all $\sum_{j=0}^k j!$ sequences of edges that a shortest path from $v_i$ to $w_j$ might cross. (These are what is called homotopy classes, I think.)}

\subsection{Geodesic versus everyone}

Next we show how to extend \cref{sameface_labelling} so that we can compute $\dist_G(v_i,z)$ for any vertex $z\in V(G)$.


The setting is the following $v$ and $w$ are vertices on the same face of $G$, $a$ is the lowest common $T$-ancestor of $v$ and $w$, $s:=\dist_G(a,v)=\depth_T(v)-\depth_T(a)$, and $\depth_T(w)-\depth_T(v)\in\{0,1\}$.  Let $v_1,\ldots,v_s:=P_T(v,a)$.  As before, we use the Jordan cycle $f$ that defines two graphs $G_0$ and $G_1$. We focus on assigning labels so that we can compute $\dist_{G_0}(v_i,x)$ for any $i\in\{1,\ldots,t\}$ and any $x\in V(G_0)$.  Assume, for now, that $G$ has no separating triangles.\todo{Deal with these later, via a separating triangle tree.}

For $x\in V(P_T(v,a))$, this is easy $\dist_{G_0}(v_i,x)=|\depth_T(x)-\depth_T(v_i)$.  Thus, a label of length $O(\log n)$ suffices.  For $x\in V(P_T(w,a))$, we can use \cref{sameface_labelling} to assign labels of length $O((\ell+s/\ell)\log n)$.  Thus, we need only consider the case where $x\in f_0$, i.e., $x\in V(G_0-P_T(v,w))$.  Let $t:=|V(G_0-P_T(v,w))|$

To do this, we want to define an $s\times t$ distance matrix that has the unit Monge property.  For each $x\in V(G_0)$ and each $i\in\{1,\ldots,s\}$, define
\[
    I_{x,i}:=\{j\in\{1,\ldots,s\}: \text{there exists a $G_0$-geodesic from $x$ to $v_i$ that contains $v_j$}\}
\]
Of course $i\in I_{x,i}$ since every $G_0$-geodesic from $x$ to $v_i$ contains $v_i$.  The following observation shows that $I_{x,i}$ is a contiguous set of integers:

\begin{obs}\label{contiguous}
    If $x\in V(G_0)$, $i\in\{1,\ldots,s\}$, and $I_{x,i}$, and integers $a <z <b$ such that $a,b\in I_{x,i}$ then $z\in I_{x,i}$.
\end{obs}

In light of \cref{contiguous}, we will define $a_{x,i}:=\min I_{x,i}$ and $b_{x,i}:=\max I_{x,i}$. The following lemma shows that $a_{x,i}$ and $b_{x,i}$ are monotone with respect to $i$.

\begin{obs}\label{monotone}
    For each $x\in V(G_0)$ and $i\in\{1,\ldots,s-1\}$, $a_{x,i} \le a_{x,i+1}$ and $b_{x,i}\le b_{x, i+1}$.
\end{obs}

Now we get to the Unit Monge Property we want:

\begin{obs}
    For each $x\in V(G_0)$ and $i\in\{1,\ldots,s-1\}$, $i\not in I_{x,i+1}$ or $i+1\not\in I_{x,i}$.
\end{obs}

\begin{proof}
    If $i\in I_{x,i+1}$, then $\dist_{G_0}(x,v_{i+1}) = \dist_{G_0}(x,v_i)+1 > \dist_{G_0}(x,v_i)$.  This implies that $\dist_{G_0}(x,v_{i+1})+\dist_{G_0}(v_{i+1},x)>\dist_{G_0}(x,v_i)$ so there is no $G_0$-geodesic from $x$ to $v_i$ that contains $v_{i+1}$, so $i+1\not\in I_{x,i}$.
\end{proof}












\section{Easy Stuff}

Let $G$ be an $n$-vertex planar graph and let $T$ be a BFS spanning tree of $G$. Define the partition $\mathcal{L}:=\{L_i: i\in\N\}$ where $L_i:=\{v\in V(G):\depth_T(v)=i\}$, for each $i\in \N$.  The partition $\mathcal{L}$ is called a \emph{BFS layering} of $G$.  Since $G$ is finite, there exists $h:=\max\{i\in\N:L_i\neq\emptyset\}$.

Let $C$ be a large constant and let let $I:=\{i\in\{1,\ldots,h\}: |L_i|\le Cn^{2/3}\}$.
Let $a:=\max\{i\in I: \sum_{i'=0}^i |L_{i'}| \le 2n/3\}$ and $b:=\min\{i\in I: \sum_{i'=i}^h |L_{i'}| \le 2n/3\}$.  We distinguish between three cases:
\begin{enumerate}
    \item $a \ge b$.  This case is easy to handle: Define the \emph{separator} $S:=L_{a}$ and observe that $G-S$ has no component larger than $2n/3$. Let $C_1,\ldots,C_k$ be the components of $G-S$.  For each $r\in\{1,\ldots,k\}$ and each $v\in V(C_r)$ the label $\varphi(v)$, of $v$ contains
    \begin{compactenum}
        \item the integer $r(v):=r$;
        \item the distance sequence $d(v):=\langle \dist(v,x): x\in L_a\rangle$; and
        \item a recursively-defined label $\varphi_1(v)$ obtained by recursing on $C_r$.
    \end{compactenum}
    To determine $\dist_G(v,w)$ given $\varphi(v)$ and $\varphi(w)$ there are two cases to consider:
    \begin{compactenum}
        \item $r(v)\neq r(w)$: In this case $S$ separates $v$ and $w$, so every path from $v$ to $w$ in $G$ contains a vertex in $S$.  In particular, there is some $(v,w)$ $G$-geodesic that contains some $x\in S$.  This implies that $\dist_G(v,w)=\min\{\dist_G(v,x)+\dist_G(w,x): x\in S\}$ and this can be computed from $d(v)$ and $d(w)$.

        \item $r:=v(v)=r(w)$.  In this case, there is a $(v,w)$ $G$-geodesic  that is contained in $C_r$, or every $(v,w)$ $G$-geodesic contains a vertex of $S$.  Therefore, $\dist_G(v,w)=\min\{\dist_{C_r}(v,w)\}\cup \{\dist_G(v,x)+\dist_G(w,x): x\in S\}$.  The first quantity in this union can be computed from $\varphi_1(v)$ and $\varphi_1(w)$. The second quantity can be computed as described above.
    \end{compactenum}

    \item $0 < b-a \le Cn^{1/3}$.  In this case, we define $S:=L_a\cup L_b\cup X$ where $X\subseteq \bigcup_{i=a+1}^{b-1}$ consists of $2$ vertical paths in $T$ such that $G-S$ has no component of size larger than $2n/3$.  The existence of such a set $X$ is implicit in the work of \citet{lipton.tarjan:applications} and is treated explicitly by \citet{dujmovic:graph}.  We then proceed exactly as in the previous case.

    \item $b-a > Cn^{1/3}$.  In this case, we let $S:=L_a\cup L_b$ and proceed as in Case~1 except that one of the components of $G-S$ contained in $G[\bigcup_{i=a+1}^{b-1}L_i]$ may have size greater than $2n/3$.  We must treat this component differently.

    Let $G_{ab}:=G[\bigcup_{i=a}^b L_i]$.  Since $|L_i|\ge Cn^{1/3}$ for each $i\in\{a+1,\ldots,b-1\}$, $b-a-1\le n^{2/3}/C$.  This is important, because it means that $G_{ab}$ has a layering using $n^{2/3}$ layers.  In particular, the separator $X$ defined in the previous case has at most  $2n^{2/3}/C$ vertices.  Using \cref{sameface_labelling}, we can therefore assign labels to vertices in $X$ of length $O(n^{1/3})$ so that we can compute $\dist_{G_{ab}}(v,w)$ for any $v,w\in X$.\todo{A bit more explanation here.}

    ``All'' that remains is to show that we can extend this to a labelling of all the vertices in $G_{ab}$ so that, for any $v,w\in G_{ab}$, we can compute $\min\{\dist_{G_{ab}}(v,x)+\dist_{G_{ab}}(x,w): x\in X\}$.  Holy shit this sounds impossible.  As a first step, let's just try to assign labels so that we can compute $\dist_{G_{ab}}(v,x)$ for any $v\in V(G_{ab})$ and any $x\in X$.
\end{enumerate}

The current plan is to show that, If $v_1,\ldots,v_k$ are vertices of $G_{ab}$ and we can't concisely represent $(\dist_{G_{ab}}(v_{k+1},x):x\in X)$ using combinations of $O(n^{1/3})$ intervals from $\{(\dist_{G_{ab}}(v_{i},x):x\in X):i\in\{1,\ldots,\}\}$ then the shortest path tree $T_{k+1}$ from $v_{k+1}$ to $X$ contains $\Omega(n^{2/3})$ edges that are not in any of the shortest path trees $T_1,\ldots,T_k$.  Let's start simple.

Let $G_{ab}'$ be one of the components of $G_{ab}-X$ and let $G_0:=G[V(G'_{ab})\cup V(X)]$.  Let $P$ and $Q$ be the two upward paths in $T$ that define $X$.  Let $M$ be the two distance matrices indexed by $V(G_{ab})\times V(P)$ and $V(G_{ab})\times Q$, respectively, where the columns of $M$ and $N$ are ordered in the same order that vertices appear in $P$ and $Q$.

We focus on $M$ and $P$, though everything we do next applies also to $N$ and $Q$.  It will be helpful for shortest paths in $G$ to be uniquely defined, and to prefer edges of $P$ over other edges in $G_0$.  To achieve both these goals, define $\epsilon := \tfrac{1}{2n^2}$, assign each edge $e$ of $P$ a generic weight $w_e$ in the interval $[1,1+\epsilon]$, and define each edge in $E(G_0)\setminus E(P)$ a generic weight $w_e$ in the interval $(1+\epsilon,1+2\epsilon)$.  In this context, generic simply means that, for any two disjoint sets $I,J\subseteq E(G_0)$, $\sum_{e\in I}w_e\neq \sum_{e\in J}w_e$.  For any two vertices $v,w$ of $G_0$, let $P_{vw}$ be the unique shortest path from $v$ to $w$ in this weighted graph.  Then
\begin{inparaenum}[(i)]
    \item for any $v,w\in V(G_0)$, $P_{vw}$ is a $(v,w)$ $G_0$ geodesic;
    \item for any $v,w\in V(G_0)$, $P_{vw}$ maximizes the number of edges of $P$ over all $(v,w)$ $G_0$-geodesics;
    \item for any $v,w,x,y\in V(G_0)$ $P_{vw}\cap P_{xy}$ is a $G_0$ geodesic.
\end{inparaenum}

Let $w$ be any vertex of $G_0$ and let $C:=M_{*,w}$ be the column vector in $M$ that lists the distances from $w$ to $v_1,\ldots,v_s:=P$, so $C_i:=M_{v_i,w}=\dist_{G_0}(v_i,w)$.   Since the edge $v_iv_{i+1}$ is in $G_0$, $C_i-C_{i+1}\in\{-1,0,1\}$, for each $i\in\{1,\ldots,s-1\}$.  Our goal is to assign labels to $V(G_0)$ so that, given the labels of $w$ and $v_i$ we can recover $\dist_{G_0}(v_i,w)=C_i$.

Let $T_w:=\bigcup_{i=1}^s P_{wv_i}$ be the shortest path tree from $w$ to $V(P)$.  Define the tree $T'_w$ by starting with $T_w$ and removing any vertex $v_i\in V(P)$ whose $T_w$-parent is also in $P$.  The number, $k$, of leaves in $T'_w$ is a useful measure of the complexity of $C_i$:

\begin{lem}
    If $T'_w$ has $k$ leaves, then $C_i$ has a representation using $O(k\log n)$ bits.
\end{lem}

\begin{proof}
    Let $I:=\{i\in\{1,\ldots,s\}: v_i\in V(T'_w)\}$. Label the elements of $I:=\{i_1,\ldots,i_k\}$ so that $i_1<i_2<\cdots<i_k$.  The representation of $C_i$ consists of the value $k$ (encoded using $O(\log n)$ bits) followed by the values $i_1,\ldots,i_k$ (encoded using $O(k\log n)$ bits), followed by the values $C_{i_1},\ldots,C_{i,k}$ (encoded using $O(k\log n)$ bits).

    For convenience, define $i_0=0$, $i_{k+1}=s+1$, and $C_{0}=C_{s+1}=\infty$.
    Given this encoding, to determine $C_i$ we find the unique $j\in\{0,\ldots,t\}$ such that $i_j\le i\le i_{j+1}$ and return $\min\{i-C_{i_j}, C_{i_{j+1}}-i\}$.
\end{proof}

% The following is not true.  We can make $T'_w$ a complete binary tree with $k$ leaves and subdivide each edge e 2^{log k - depth(e)} times.  The resulting tree has only $O(k\log k)$ nodes.
% The number of leaves of $T'_w$ also gives a surprisingly large lower bound on the size of $T'_w$:
%
% \begin{lem}
%     If $T'_w$ has $k$ leaves then $|T'_w)|\in \Omega(k^2)$.
% \end{lem}

\begin{lem}
    For any $w_1,w_2\in V(G_0)$ with $d:=\dist_{G_0}(w_1,w_2)$, there exists a cyclically unimodal\todo{Define cyclically unimodal} $s$-vector $v\in\{-d,\ldots,d\}^s$ such that $M_{*,w_1}=M_{*,w_2}+v$.
\end{lem}

At this point, I'm well and truly stuck.  I was hoping to maybe do something with ideas from computational geometry related to ham-sandwiches, and/or centerpoints, but have not had much progress. For every $v_i\in V(P)$ and $w\in V(G_0)$ there is a collection of $G_0$ geodesics that contain $v_i$ and $w$.  (Start with the shortest path tree  $T_{v_i}$ and keep only the vertices that have $w$ as a descendant or an ancestor.)

These things behave a bit like lines. For example, for every $v_i$ there is a "halving line" that contains $v_i$ (this is a root-to-leaf path in $T_{v_i}$ with at most half the tree to its left and half the tree to its right.)  The vertices in $G_0$ also behave a bit like they have some size.  (Think of $v_i$ as a light and the subtree of $T_{v_i}$ rooted at $w$ as the shadow of $w$; as $w$ moves further away from $v_i$ it's shadow gets smaller.)

Let $L(v_i,w)$ be the set of nodes in the subtree of $T_{v_i}$ that are related to $w$.  It seems like it would be useful to find $w$, $i$, and $\ell$ $G_0$ such that $\bigcap_{j=i}^{i+\ell} v_j L(v_j,w)$ has size $k$ for some large value of $k$.  This would allows us to encode all $k\ell$ distances $\dist_{G_0}(v_j,w')$ by having each of the nodes involved store their distances to $w$.

\section{An $\tilde{O}(\sqrt{n})$ bound using crossing numbers}

Here's an alternative strategy.  Let $M$ be a set of unordered pairs of vertices of $G$ and let $G_M$ be the graph with vertex set $V(G)$ and edge set $E(G_M)=M$.  We obtain drawing of $G_M$ by drawing each edge $vw$ using (a slight perturbation of) the shortest path from $v$ to $w$ in $G$.  By the crossing lemma, this drawing has $\Omega(|M|^3/n^2)$ pairs of crossings edges.  Each pair of crossing edges corresponds to a pair of shortest paths in $G$ with a common vertex and we can charge that pair to their common vertex.  When we do this, some vertex $v\in V(G)$ is charged $\Omega((|M|/n)^3)$ times.  This means $v$ is involved in this many crossing pairs and therefore there are at least $\Omega((|M|/n)^{3/2})$ pairs $(x,y)\in M$ such that the shortest path from $x$ to $y$ in $G$ contains $v$.  This means that there are $\Omega((|M|/n)^{3/2})$ entries in the distance matrix that can be stored by having each vertex remember its distance to $v$.

Let $M_0$ consist of the $|M_0|=\binom{n}{2}$ pairs of vertices in $G$.  Applying the above, we have each vertex remember its distance to $v$ and this takes care of $\Omega((|M_0|/n)^{3/2})=\Omega(n^{3/2})$ entries in the distance matrix.  We then form $M_1$ by removing these pairs from $M_0$ and repeat.  Let $m_k=|M_k|$, this leads to a recurrence of the form:
\[  m_k = m_{k-1} - (m_{k-1}/n)^{3/2} \]
with $m_0=\binom{n}{2}$.  This starts out well and continues to do so until $k=\sqrt{n}$, at which point $m_k\approx \tilde{O}(n^{1+\epsilon})$.  Then it becomes much less efficient.  Still, it's hopeful.

We may even be able to do this without using the crossing lemma.  If we could get down to a case where the average distance between a of vertices $v,w$ of $G$ is $\Omega(\sqrt{n})$ then the total length of all paths is $\Omega(n^{5/2})$, so some vertex is involved in $\Omega(n^{3/2})$ paths.

Hold on now.  We already know what to do if we have a BFS layering in which every $n^{1/3}$ consecutive layers contains a layer of size at most $n^{1/3}$. We also know what to if we have a separator of size $n^{1/3}$.  Thus, we can assume that for each $v$ in $G$, the BFS layering of $v$ contains a set of $\Omega(n^{1/3})$ consecutive layers each of size at least $n^{1/3}$ and whose total size is at least $(1-\epsilon)n$.  If I could show that the distribution over these layers is fairly uniform, then I would get an average distance of $\Omega(n^{1/3})$.

More usefully, in the product $G\subseteq H\boxtimes P$, $\diam(G)\ge\max\{\diam(P),\diam(H)\}$.  We already know $\diam(P)$ is large.  I think we can also figure out that $\diam(H)$ is large since, otherwise, we get a small separator.  So both $\diam(P)$ and $\diam(H)$ are large.

 is separated by at most $n^{1/3}$ layers of size greater than $n^{1/3}$, so we can assume we don't have one of those.  Therefore, for every $v\in V(G)$, the BFS layering of $G$ contains a


Can we use this kind of approach to beat $O(\sqrt{n})$-length labels?  One very slack part of this argument comes from the fact that we only charge each pair of crossing paths for one of the vertices they have in common. I would guess that, on average, they have $\omega(1)$ vertices in common.  If an average pair had $\Omega(n^{1/3})$ vertices in common then we would get
\[  m_1 = \binom{n}{2} - \Omega(n^{5/3}) \enspace . \]
This is a great start, since it could lead to be done when $k=O(n^{1/3})$.














\bibliographystyle{plainurlnat}
\bibliography{distance-labelling}


\end{document}
